% Tipo
\documentclass{article}

% Márgen
\usepackage[margin = 1.5cm]{geometry}

% Símbolos
\usepackage{amsmath}

%Imágenes
\usepackage{graphicx}

%Idioma
\usepackage[spanish]{babel}
\usepackage[utf8]{inputenc}

\begin{document}
    \title{
        Organización y Arquitectura de Computadoras \\
        2019-2 \\
        Práctica 5: Lógica Secuencial
    }
    \author{
        Sandra del Mar Soto Corderi \\
        Edgar Quiroz Castañeda
    }
    \date{
        13 de marzo del 2019
    }

    \maketitle

    \section{Preguntas}

    \begin{enumerate}
        %1
        \item 
            ¿En qué difieren los distintos tipos de flip-flops?\\
            
            
            ¿Cómo se decide qué tipo se usará en el circuito?\\
            

        %2		
		\item
        Un registro de desplazamiento es un circuito secuencial que desplaza a la izquierda o a la derecha la información contenida en el. Considerando el desplazamiento de 1 bit a la izquierda, ¿cómo se implementa dicho circuito?\\
        
        ¿Cómo podríamos simular su funcionamiento con las operaciones que se tienen en la ALU de 8 bits? \\
      
                 
    \end{enumerate}
  
    
    
\end{document}