% Tipo
\documentclass{article}

% Márgen
\usepackage[margin = 1.5cm]{geometry}

% Símbolos
\usepackage{amsmath}
\usepackage{amssymb}

%Imágenes y tablas
\usepackage{graphicx}
\usepackage[table,xcdraw]{xcolor}
\usepackage{float}

% Mapas de Karnaugh
\usepackage{karnaugh-map}

%Idioma
\usepackage[spanish]{babel}
\usepackage[utf8]{inputenc}

\begin{document}
    \title{
        Organización y Arquitectura de Computadoras \\
        2019-2 \\
        Práctica 5: Lógica Secuencial
    }
    \author{
        Sandra del Mar Soto Corderi \\
        Edgar Quiroz Castañeda
    }
    \date{
        13 de marzo del 2019
    }

    \maketitle
    
    \section{Minimización del autómata}
    
 		La tabla de verdad del autómata se vería como sigue:
    	\begin{table}[H]
    		\caption{Tabla de verdad del autómata}
    		\begin{center}
    		\begin{tabular}{|l|ll|ll|l|l|l|l|}
    			\hline
    			\cellcolor[HTML]{FFCCC9}In & \cellcolor[HTML]{FFFFC7}{\color[HTML]{000000} E} & \cellcolor[HTML]{FFFFC7}{\color[HTML]{000000} } & \cellcolor[HTML]{FFCC67}$E_{+1}$ & \cellcolor[HTML]{FFCC67} & \cellcolor[HTML]{38FFF8}$S_{1}$ & \cellcolor[HTML]{96FFFB}$R_{1}$ & \cellcolor[HTML]{B979FF}$S_{2}$ & \cellcolor[HTML]{D7B1FF}{\color[HTML]{000000} $R_{2}$} \\ \hline
    			0                          & \multicolumn{1}{l|}{0}                           & 0                                               & \multicolumn{1}{l|}{0}           & 1                        & 0                               & X                               & 1                               & 0                                                      \\ \hline
    			1                          & \multicolumn{1}{l|}{0}                           & 0                                               & \multicolumn{1}{l|}{0}           & 0                        & 0                               & X                               & 0                               & X                                                      \\ \hline
    			0                          & \multicolumn{1}{l|}{0}                           & 1                                               & \multicolumn{1}{l|}{1}           & 0                        & 1                               & 0                               & 0                               & 1                                                      \\ \hline
    			1                          & \multicolumn{1}{l|}{0}                           & 1                                               & \multicolumn{1}{l|}{0}           & 0                        & 0                               & X                               & 0                               & 1                                                      \\ \hline
    			0                          & \multicolumn{1}{l|}{1}                           & 0                                               & \multicolumn{1}{l|}{1}           & 1                        & X                               & 0                               & 1                               & 0                                                      \\ \hline
    			1                          & \multicolumn{1}{l|}{1}                           & 0                                               & \multicolumn{1}{l|}{1}           & 0                        & X                               & 0                               & 0                               & X                                                      \\ \hline
    			0                          & \multicolumn{1}{l|}{1}                           & 1                                               & \multicolumn{1}{l|}{0}           & 0                        & 0                               & 1                               & 0                               & 1                                                      \\ \hline
    			1                          & \multicolumn{1}{l|}{1}                           & 1                                               & \multicolumn{1}{l|}{1}           & 0                        & X                               & 0                               & 0                               & 1                                                      \\ \hline
    		\end{tabular}
    	\end{center}
    	\end{table}

    
    
    \begin{itemize}
    	\item {
    	$S_{1} =  \quad \overline{Q_{1}} Q_{2} \quad \overline{In}$
    	\begin{center}
    		\begin{karnaugh-map}[4][2][1][$Q_{1}Q_{2}$][$In$]
    			\minterms{1}
    			\maxterms{0,3,4,5}
    			\terms{2}{X}
    			\terms{6}{X}
    			\terms{7}{X}
    			\implicant{1}{1}
    		\end{karnaugh-map}
    	\end{center}	
    	}
    \item {
    	$S_{2} =  \quad \overline{Q_{2}} \quad \overline{In}$
    	\begin{center}
    		\begin{karnaugh-map}[4][2][1][$Q_{1}Q_{2}$][$In$]
    			\minterms{0,2}
    			\maxterms{1,3,4,5,6,7}
    			\implicantedge{0}{0}{2}{2}
    		\end{karnaugh-map}
    	\end{center}	
    }
\item {
	$R_{1} =  Q_{1} Q_{2} \quad \overline{In}$
	\begin{center}
		\begin{karnaugh-map}[4][2][1][$Q_{1}Q_{2}$][$In$]
			\minterms{3}
			\maxterms{1,2,6,7}
			\terms{0}{X}
			\terms{4}{X}
			\terms{5}{X}
			\implicant{3}{3}
		\end{karnaugh-map}
	\end{center}	
}
\item {
	$R_{2} =  Q_{2}$
	\begin{center}
		\begin{karnaugh-map}[4][2][1][$Q_{1}Q_{2}$][$In$]
			\minterms{1,3,5,7}
			\maxterms{0,2}
			\terms{4}{X}
			\terms{6}{X}
			\implicant{1}{7}
		\end{karnaugh-map}
	\end{center}	
}
    
    \end{itemize}
    
     

    \section{Preguntas}

    \begin{enumerate}
        %1
        \item 
            ¿En qué difieren los distintos tipos de flip-flops?\\
            En cómo están configuradas sus entradas. Esto surge de que al tener 
            entradadas de $set$ y $reset$ separadas (que son operaciones 
            opuestas) no puedes garantizar que no vayan en entrar 
            en conflicto. \\
            El flip-flop $SR$ simplemente asigna como \textit{indefinido} cuando 
            ambos estén encendidos, y sin cambios cuando ambos estén apagados. \\
            El flip-flop $JK$ es igual, sólo que cuando abmos están encendidos, 
            se invierte el valor actual del estado.\\
            Y el flip-flop $D$ quita ambas entradas y las remplaza con una sólo, 
            que cuando está encendida marca que hay que invertir el estado 
            actual, y cuando está apagado indica que no va a cambiar nada.
            
            
            ¿Cómo se decide qué tipo se usará en el circuito?\\
            Al final, los tres tienen un comportamiento equivalente de almacenar
            y modificar un bit, por lo que todo lo que puedes hacer con uno se 
            puede hacer con los otro.\\
            Pero dependiendo de la aplicación específica, es posible que 
            los circuitos sean más pequeños usando cierto tipo. Esto es porque
            los diferentes flip-flops tienen diferentes ``valores comodín'' en 
            sus tablas de transición que se pueden usar a favor al minimizar los 
            circuitos deseados.
            

        %2		
		\item
            Un registro de desplazamiento es un circuito secuencial que 
            desplaza a la izquierda o a la derecha la información contenida en 
            el. 
            Considerando el desplazamiento de 1 bit a la izquierda, ¿cómo se 
            implementa dicho circuito?\\
            De forma abstracta, desplazar todo a la izquierda un bit significa 
            agregar un bit a la derecha, que significa que todo se multiplicó 
            por la base, es decir por dos. Esto es análogo al sistema decimal, 
            donde agregar un cero a la derecha equivale a multiplicar el número 
            por 10, que es la base.
        
            ¿Cómo podríamos simular su funcionamiento con las operaciones que se 
            tienen en la ALU de 8 bits? \\
            Multiplicar un número por $n \in \mathbb{N}$ es equivalente a sumar ese 
            número consigo mismo $n-1$ veces. \\
            Entonces, para multiplicarlo por 2 basta sumar el número consigo 
            mismo una vez. \\
            Y la suma ya está implemetada en la ALU de 8 bits.
      
                 
    \end{enumerate}
  
    
    
\end{document}